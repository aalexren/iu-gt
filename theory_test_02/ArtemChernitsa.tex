%%%%%%%%%%%%%%%%%%%%%%%%%%%%%%%%%%%%%%%%%
% University/School Laboratory Report
% LaTeX Template
% Version 4.0 (March 21, 2022)
%
% This template originates from:
% https://www.LaTeXTemplates.com
%
% Authors:
% Vel (vel@latextemplates.com)
% Linux and Unix Users Group at Virginia Tech Wiki
%
% License:
% CC BY-NC-SA 4.0 (https://creativecommons.org/licenses/by-nc-sa/4.0/)
%
%%%%%%%%%%%%%%%%%%%%%%%%%%%%%%%%%%%%%%%%%

%----------------------------------------------------------------------------------------
%	PACKAGES AND DOCUMENT CONFIGURATIONS
%----------------------------------------------------------------------------------------

\documentclass[
	a4paper, % Paper size, specify a4paper (A4) or letterpaper (US letter)
	11pt, % Default font size, specify 10pt, 11pt or 12pt
]{CSUniSchoolLabReport}

\addbibresource{sample.bib} % Bibliography file (located in the same folder as the template)

%----------------------------------------------------------------------------------------
%	REPORT INFORMATION
%----------------------------------------------------------------------------------------

\title{FINAL EXAMINATION \\ Report \\ Game Theory} % Report title

\author{Artem \textsc{Chernitsa}} % Author name(s), add additional authors like: '\& James \textsc{Smith}'

\date{\today} % Date of the report

%----------------------------------------------------------------------------------------

\begin{document}

\maketitle % Insert the title, author and date using the information specified above

\begin{center}
	\begin{tabular}{l r}
		Date Performed: & December 13, 2022 \\ % Date the experiment was performed
		% Partners: & Cecilia \textsc{Smith} \\ % Partner names
		% & Tajel \textsc{Khumalo} \\
		Instructor: & Professor \textsc{Shilov} % Instructor/supervisor
	\end{tabular}
\end{center}

% If you need to include an abstract, uncomment the lines below
\begin{abstract}
    It is a distance asynchronous individual written test to check that students understand and can apply main definitions, concepts and techniques presented on lectures (topics 3 and 4 presented on weeks 7-11 and 12-14). Individual data for tasks (according to birthdate 09.03.2000): $day = 09, month = 03, year = 2000$.
\end{abstract}

%----------------------------------------------------------------------------------------
%	TASK 1
%----------------------------------------------------------------------------------------

\section{Task 1}


% To determine the atomic weight of magnesium via its reaction with oxygen and to study the stoichiometry of the reaction (as defined in \ref{definitions}):

% \begin{center}
% 	\ce{2 Mg + O2 -> 2 MgO} % Chemical equations entered in \ce{} commands, see the mhchem package documentation for more information
% \end{center}

% If you have more than one objective, uncomment the below:
% \begin{description}
% 	\item[First Objective] \hfill \\
% 	Objective 1 text
%	\item[Second Objective] \hfill \\
%	Objective 2 text
% \end{description}

% \subsection{Definitions}\label{definitions} % Labels provide a point for referencing, in this case with \ref{definitions} to refer to this subsection number
\subsection{Problem Description}\label{writeup01}

% \begin{description}
    Consider a game of two players (Alice and Bob) with the following payoff matrix
    $\begin{bmatrix}
        day & 24 & ye & 19 \\
        month & 4 & ar & 61
    \end{bmatrix}
    $. Rows of the matrix corresponds to strategies $A1$ and $A2$ of Alice, columns - to strategies $B1, B2, B3, B4$ of Bob.
    Firstly, characterize the game using terms and concepts introduced in the lecture notes. Then solve the game in mixed strategies.

	% \item[Stoichiometry] The relationship between the relative quantities of substances taking part in a reaction or forming a compound, typically a ratio of whole integers.
	% \item[Atomic mass] The mass of an atom of a chemical element expressed in atomic mass units. It is approximately equivalent to the number of protons and neutrons in the atom (the mass number) or to the average number allowing for the relative abundances of different isotopes. 
% \end{description} 

\subsection{Solution}\label{solution01}

\begin{description}
    \item[Two player's game in normal form (topic 2, page 14th):] Two players' game in the normal form is the 2-dimensional finite table (matrix) where:
    \begin{itemize}
        \item rows correspond to the strategies of the first player,
        \item columns correspond to the strategies of the second player,
        \item each row and each column produce a play and their intersection
cell contains vector of individual payoffs.
    \end{itemize}
    \item[Payoff (topic 2, page 16th):] Let $\pi$ be payoff function that maps every play $S = (s_A, s_B, ...)$ of the game, where $s_A, s_B, ...$ are strategies of players $A, B, ...$ into vector of the payoffs in this play $\pi(S) = (\pi_A(S):\pi_B(S):...)$.
    \item[Zero-sum game in normal form (topic 2, page 26-27th):] A game in the normal form is zero-sum if for every play $S$ the overall sum $\sum_{X \text{ is a player}}{\pi(X)} = 0$. It also called matrix game, because payoffs can be represented by a matrix with the payoffs of the first player.
    \item[Solve the game in normal form (according to prof. Shilov):] Solving the game in normal form means finding all its Nash equilibria - in pure and mixed strategies.
    \item[Solve the game (in normal form) in pure strategies (topic 2, page 36th):] To solve in pure strategies a game (in the normal form) means to find all Nash equilibria of the game.
    \item[Nash equilibrium (topic 2, page 17th):] Nash equilibrium is any play that is acceptable for all players.
    \item[Acceptable play (topic 2, page 17th):] A play $S = (s_A, s_B, ...)$ is acceptable for a player $X$ in $A, B, ...$ if $\pi_X(S) \geq \pi_X(S_{X:s'_X})$ for any legal strategy $s'_X$ of the player X.
    \item[Notation (topic 2, page 16th):] For any play $S = (s_A, s_B, ...)$ and any strategy $s'_X$ of a player $X$ in $A, B, ...$, let $S_{X:s'_X}$ be result of $X$ playing $s'_X$ instead of $s_X$ in S (while all other players do not change their strategies).
    \item[Strict domination (topic 2, page 43th):] For any player $X$ in $(A, B, ...)$, for any two strategies $s'_X$ and $s''_X$ let us say that $s'_X$ strictly dominates $s''_X$ if $\pi_X(S_{X:s'_X}) >  \pi_X(S_{X:s''_X})$ for every play $S$.
    \item[Elimination of dominated strategy (topic 2, page 44th):] Let $G$ be a game in the normal form of players $A, B, ...$, $X$ be a player in $A, B, ...$, with a strategy $s_X$ dominated by some another strategy of the player. Then a game $G(X\backslash s_X)$ resulting from $G$ by elimination (prohibition) of the strategy $s_X$ for $X$ has the same set of Nash equilibria as $G$.
    \item[Probabilities (topic 3, page 5th):] Every pure strategy is a mixed strategy in which this pure strategy is chosen with probability 1, and all others – with probability 0.
\end{description}\hspace{2pt}

Thus it could be represented as zero-sum game of two players in the normal form. Let first player will be A, and the second is B.
According to the birthdate matrix should be look like this:
$$
\begin{bmatrix}
    9:-9 & 24:-24 & 20:-20 & 19:-19 \\
    3:-3 & 4:-4 & 0:0 & 61:-61
\end{bmatrix}
$$
At the beginning I need to eliminate the dominated strategies:\\
\begin{itemize}
    \item strategy $B1$ dominates $B2$ because $\pi_B(A1, B1) > \pi_B(A1, B2) = -9 > -24$ and $\pi_B(A2, B1) > \pi_B(A2, B2) = -3 > -4$.
    \item strategy $B1$ dominates $B4$ because $\pi_B(A1, B1) > \pi_B(A1, B4) = -9 > -19$ and $\pi_B(A2, B1) > \pi_B(A2, B4) = -9 > -61$.
\end{itemize}\hspace{2pt}

Hence I got the following matrix:
$$
\begin{bmatrix}
    9:-9 & 20:-20 \\
    3:-3 & 0:0
\end{bmatrix}
$$
Now let's check for the presence of Nash equilibrium positions in pure strategies, since each Nash equilibrium in pure strategies is also an equilibrium in mixed strategies (according to topic 3, page 31th). To find it I should check if matrix follows minmax property (topic 2, page 35th): $\underset{S_B}{min}\, \underset{S_A}{max}\, G_{S_AS_B} = \underset{S_A}{max}\, \underset{S_B}{min}\, G_{S_AS_B}$. Also, let's present matrix in shortened form: 
$$
\begin{bmatrix}
    9 & 20 \\
    3 & 0
\end{bmatrix}
$$
So, $\underset{S_B}{min}\, \underset{S_A}{max}\, G_{S_AS_B} = 9$, $\underset{S_A}{max}\, \underset{S_B}{min}\, G_{S_AS_B} = 9$. As I see there is Nash equilibrium $(A1,B1)$, meaning I can skip checking it in mixed strategies.

\subsection{Answer}\label{answer01}
\begin{itemize}
    \item The game could be characterized as zero-sum game in the normal form of two players.
    \item Solution of the game is Nash equilibria, that could be presented as play $S$ in mixed strategies: $((1, 0)(1, 0, 0, 0))$. Where play $S$ in mixed strategies stands for $((a_1, 0)(b_1, 0, 0, 0))$, where $a_1$ - probability for player A to choose strategy $A1$, $b_1$ - probability for player B to choose strategy $B1$.
\end{itemize}

%----------------------------------------------------------------------------------------
%	TASK 2
%----------------------------------------------------------------------------------------
\section{Task 2}

\subsection{Problem Description}\label{writeup02}
Consider a game of two players (Alice and Bob) with the following payoff matrix
$\begin{bmatrix}
    day:24 & ye:19 \\
    month:4 & ar:61
\end{bmatrix}
$. Rows of the matrix corresponds to strategies $A1$ and $A2$ of Alice, columns - to strategies $B1$ and $B2$ of Bob. Firstly, characterize the game using terms and concepts introduced in the lecture notes. Then solve the game in mixed strategies.

\subsection{Solution}\label{solution02}
As was mentioned in references \ref{solution01} this game could be characterized as game in normal form of two players. Let's fill matrix according to birthdate:
$$
\begin{bmatrix}
    9:24 & 20:19 \\
    3:4 & 0:61
\end{bmatrix}
$$
Define Alice as first player $A$, and the Bob as the second player $B$. Next, eliminate dominated strategies (according to \ref{solution01}):
\begin{itemize}
    \item $A1$ dominates $A2$ since $\pi_A(A1, B1) = 9 > \pi_A(A2, B1) = 3$ and $\pi_A(A1, B2) = 20 > \pi_A(A2, B2) = 0$
\end{itemize}\hspace{2pt}

I got
$$
\begin{bmatrix}
    9:24 & 20:19
\end{bmatrix}
$$
\begin{itemize}
    \item $B1$ dominates $B2$ since $\pi_B(A1, B1) = 24 > \pi_B(A1, B2) = 19$
\end{itemize}\hspace{2pt}

Therefore the final matrix is
$$
\begin{bmatrix}
    9:24
\end{bmatrix}
$$
Meaning I found Nash equilibrium, and it will be solution of the game in pure strategies (similar to \ref{solution01}).

\subsection{Answer}\label{answer02}
\begin{itemize}
    \item The game could be characterized as game of two players in normal form.
    \item Solution of the game is Nash equilibria, that could be presented as play $S$ in mixed strategies: $((1, 0)(1, 0))$. Where play $S$ in mixed strategies stands for $((a_1, 0)(b_1, 0))$, where $a_1$ - probability for player $A$ to choose strategy $A1$, $b_1$ - probability for player $B$ to choose strategy $B1$.
\end{itemize}
% \begin{tabular}{l l}
% 	Mass of empty crucible & \SI{7.28}{\gram}\\ % Scientific/technical units are output using the \SI command, see the siunitx package documentation for more information on how to use this command
% 	Mass of crucible and magnesium before heating & \SI{8.59}{\gram}\\
% 	Mass of crucible and magnesium oxide after heating & \SI{9.46}{\gram}\\
% 	Balance used & \#4\\
% 	Magnesium from sample bottle & \#1
% \end{tabular}


%----------------------------------------------------------------------------------------
%	TASK 3
%----------------------------------------------------------------------------------------
\section{Task 3}

\subsection{Problem Description}\label{writeup03}
Consider problem Rational Agents at the Marketplace (from lecture notes on topic 4). What are individual agents’ beliefs, desires, and intentions in the model of the problem? Let agents $A$ and $B$ compete for a salesman, and the matrix of their game flip-or-bid game be
$$
\begin{bmatrix}
    A\backslash B & bid & flip \\
    bid & -ye:-ar & 0:-month \\
    flip & -day:0 & -day:-month
\end{bmatrix}
$$
where $L_A = -day$ and $L_B = -month$ are individual (negative) losses in case of flip, $F_A = -ye$ and $F_B = -ar$ are individual (also negative) fins for simultaneous bidding. Characterize and solve flip-or-bid game.

\subsection{Solution}\label{solution03}
This game could be characterized as game of two players in normal form (w.r.t. \ref{solution01}). Rows are strategies for player $A$, and columns are strategies for player $B$, thus organizing plays with individual payoffs. According to birthdate matrix is following:
$$
\begin{bmatrix}
    -20:0 & 0:-3 \\
    -9:0 & -9:-3
\end{bmatrix}
$$
As in \ref{solution01} and \ref{solution02} eliminate dominated strategies to find all Nash equilibria:
\begin{itemize}
    \item $B1$ dominates $B2$ since $\pi_B(A1, B1) = 0 > \pi_B(A1, B2) = -3$ and $\pi_B(A2, B1) = 0 > \pi(A2, B2) = -3$
\end{itemize}\hspace{2pt}

Hence matrix will be
$$
\begin{bmatrix}
    -20:0 \\
    -9:0
\end{bmatrix}
$$
Continue elimination:
\begin{itemize}
    \item $A2$ dominates $A1$ since $\pi_A(A2,B1) = -9 > \pi_A(A1, B1) = -20$
\end{itemize}\hspace{2pt}

So, it should look like this:
$$
\begin{bmatrix}
    -9:0
\end{bmatrix}
$$
Meaning that $(A2, B1)$ is Nash equilibrium (w.r.t. \ref{solution01}). This is solution in pure strategies, and could be presented as solution in mixed strategies as $((0, 1)(1, 0))$.

\subsection{Answer}\label{answer03}
\begin{itemize}
    \item Beliefs (w.r.t topic 4, page 7th, 17-21th):
    \begin{itemize}
        \item All buyers are rational agents that can communicate, negotiate, make concessions, and flip individually and swap their salesmen pairwise (and only pairwise) in peer-to-peer manner. Every agent would like to communicate with any other will communicate eventually.
    \end{itemize}
    \item Desires (w.r.t topic 4, page 7th):
    \begin{itemize}
        \item Every buyer want to obtain exactly one cake with optimal price.
    \end{itemize}
    \item Intentions (w.r.t topic 4, page 7th):
    \begin{itemize}
        \item Every buyer can flip or bid salesman, so according to price.
    \end{itemize}
    \item Game could be characterized as game in normal form of two players.
    \item Solution of the game is Nash equilibria, that could be presented as play in mixed strategies (as in \ref{answer01} and \ref{answer02}): $((0, 1)(1, 0))$.
\end{itemize}
% \section{Sample Calculation}

% \begin{tabular}{ll}
% 	Mass of magnesium metal & = \SI{8.59}{\gram} - \SI{7.28}{\gram}\\
% 	& = \SI{1.31}{\gram}\\
% 	Mass of magnesium oxide & = \SI{9.46}{\gram} - \SI{7.28}{\gram}\\
% 	& = \SI{2.18}{\gram}\\
% 	Mass of oxygen & = \SI{2.18}{\gram} - \SI{1.31}{\gram}\\
% 	& = \SI{0.87}{\gram}
% \end{tabular}

% Because of this reaction, the required ratio is the atomic weight of magnesium: \SI{16.00}{\gram} of oxygen as experimental mass of Mg: experimental mass of oxygen or $\frac{x}{1.31} = \frac{16}{0.87}$ from which, $M_{\ce{Mg}} = 16.00 \times \frac{1.31}{0.87} = 24.1 = \SI{24}{\gram\per\mole}$ (to two significant figures).

%----------------------------------------------------------------------------------------
%	RESULTS AND CONCLUSIONS
%----------------------------------------------------------------------------------------

% \section{Results and Conclusions}

% The atomic weight of magnesium is concluded to be \SI{24}{\gram\per\mol}, as determined by the stoichiometry of its chemical combination with oxygen. This result is in agreement with the accepted value.

% \begin{figure}[H] % [H] forces the figure to be placed exactly where it appears in the text
% 	\centering % Horizontally center the figure
% 	\includegraphics[width=0.65\textwidth]{placeholder} % Include the figure
% 	\caption{Figure caption.}
% \end{figure}

% %----------------------------------------------------------------------------------------
% %	DISCUSSION
% %----------------------------------------------------------------------------------------

% \section{Discussion of Experimental Uncertainty}

% The accepted value (periodic table) is \SI{24.3}{\gram\per\mole} \autocite{Smith:2022qr}. The percentage discrepancy between the accepted value and the result obtained here is 1.3\%. Because only a single measurement was made, it is not possible to calculate an estimated standard deviation (see \textcite{Smith:2021jd}).

% The most obvious source of experimental uncertainty is the limited precision of the balance. Other potential sources of experimental uncertainty are: the reaction might not be complete; if not enough time was allowed for total oxidation, less than complete oxidation of the magnesium might have, in part, reacted with nitrogen in the air (incorrect reaction); the magnesium oxide might have absorbed water from the air, and thus weigh ``too much." Because the result obtained is close to the accepted value it is possible that some of these experimental uncertainties have fortuitously cancelled one another.

% %----------------------------------------------------------------------------------------
% %	ANSWERS TO DEFINITIONS
% %----------------------------------------------------------------------------------------

% \section{Answers to Definitions}

% \begin{enumerate}
% 	\item The \textit{atomic weight of an element} is the relative weight of one of its atoms compared to C-12 with a weight of 12.0000000$\ldots$, hydrogen with a weight of 1.008, to oxygen with a weight of 16.00. Atomic weight is also the average weight of all the atoms of that element as they occur in nature.
% 	\item The \textit{units of atomic weight} are two-fold, with an identical numerical value. They are g/mole of atoms (or just g/mol) or amu/atom.
% 	\item \textit{Percentage discrepancy} between an accepted (literature) value and an experimental value is:
% 		\begin{equation*}
% 			\frac{\mathrm{experimental\;result} - \mathrm{accepted\;result}}{\mathrm{accepted\;result}}
% 		\end{equation*}
% \end{enumerate}

%----------------------------------------------------------------------------------------
%	BIBLIOGRAPHY
%----------------------------------------------------------------------------------------

% \printbibliography % Output the bibliography

%----------------------------------------------------------------------------------------

\end{document}
